% arara: xelatex
\documentclass[12pt]{article}

\usepackage{physics}


\usepackage{tikz} % картинки в tikz
\usepackage{microtype} % свешивание пунктуации

\usepackage{array} % для столбцов фиксированной ширины

\usepackage{indentfirst} % отступ в первом параграфе

\usepackage{sectsty} % для центрирования названий частей
\allsectionsfont{\centering}

\usepackage{amsmath, amsfonts, amssymb} % куча стандартных математических плюшек

\usepackage{comment}

\usepackage[top=2cm, left=1.2cm, right=1.2cm, bottom=2cm]{geometry} % размер текста на странице

\usepackage{lastpage} % чтобы узнать номер последней страницы

\usepackage{enumitem} % дополнительные плюшки для списков
%  например \begin{enumerate}[resume] позволяет продолжить нумерацию в новом списке
\usepackage{caption}

\usepackage{url} % to use \url{link to web}

\usepackage{fancyhdr} % весёлые колонтитулы
\pagestyle{fancy}
\lhead{DSE, ICEF, Fall 2023}
\chead{}
\rhead{Midterm}
\lfoot{}
\cfoot{DON'T PANIC}
\rfoot{\thepage/\pageref{LastPage}}
\renewcommand{\headrulewidth}{0.4pt}
\renewcommand{\footrulewidth}{0.4pt}

\usepackage{tcolorbox} % рамочки!

\usepackage{todonotes} % для вставки в документ заметок о том, что осталось сделать
% \todo{Здесь надо коэффициенты исправить}
% \missingfigure{Здесь будет Последний день Помпеи}
% \listoftodos - печатает все поставленные \todo'шки


% более красивые таблицы
\usepackage{booktabs}
% заповеди из докупентации:
% 1. Не используйте вертикальные линни
% 2. Не используйте двойные линии
% 3. Единицы измерения - в шапку таблицы
% 4. Не сокращайте .1 вместо 0.1
% 5. Повторяющееся значение повторяйте, а не говорите "то же"



\usepackage{fontspec}
\usepackage{polyglossia}

\setmainlanguage{english}
\setotherlanguages{english}

% download "Linux Libertine" fonts:
% http://www.linuxlibertine.org/index.php?id=91&L=1
\setmainfont{Linux Libertine O} % or Helvetica, Arial, Cambria
% why do we need \newfontfamily:
% http://tex.stackexchange.com/questions/91507/
\newfontfamily{\cyrillicfonttt}{Linux Libertine O}

%\AddEnumerateCounter{\asbuk}{\russian@alph}{щ} % для списков с русскими буквами
%\setlist[enumerate, 2]{label=\asbuk*),ref=\asbuk*}

%% эконометрические сокращения
\DeclareMathOperator{\Cov}{\mathbb{C}ov}
\DeclareMathOperator{\Corr}{\mathbb{C}orr}
\DeclareMathOperator{\Var}{\mathbb{V}ar}

\let\P\relax
\DeclareMathOperator{\P}{\mathbb{P}}

\DeclareMathOperator{\E}{\mathbb{E}}
% \DeclareMathOperator{\tr}{trace}
\DeclareMathOperator{\card}{card}
\DeclareMathOperator{\plim}{plim}
\DeclareMathOperator{\pCorr}{\mathrm{p}\mathbb{C}\mathrm{orr}}


\newcommand \hb{\hat{\beta}}
\newcommand \hs{\hat{\sigma}}
\newcommand \htheta{\hat{\theta}}
\newcommand \s{\sigma}
\newcommand \hy{\hat{y}}
\newcommand \hY{\hat{Y}}
\newcommand \e{\varepsilon}
\newcommand \he{\hat{\e}}
\newcommand \z{z}
\newcommand \hVar{\widehat{\Var}}
\newcommand \hCorr{\widehat{\Corr}}
\newcommand \hCov{\widehat{\Cov}}
\newcommand \cN{\mathcal{N}}
\newcommand \RR{\mathbb{R}}
\newcommand \NN{\mathbb{N}}
\newcommand{\cF}{\mathcal{F}}
\newcommand{\cH}{\mathcal{H}}
\newcommand{\dBin}{\mathrm{Bin}}


\begin{document}



\begin{enumerate}

\item Let $p$ be the equiprobable distribution on natural numbers from 1 to 3, 
and $q$ be the equiprobable distribution on natural numbers from 1 to 6.

Calculate $CE(q||p)$ and $CE(p||q)$.


\item The random variable $Y_i$ are independent identically distributed and take 
the values $1$, $2$ and $3$ with unknown probabilities $p_1 + p_2 + p_3 = 1$.

You have a sample of 1000 observations with $\bar y = 2$. 
The law of large numbers allows you to assume that $\E(Y_i) = 2$.

Estimate $\hat p_1$, $\hat p_2$ and $\hat p_3$ if you believe in the most unpredictable world given observed constraints. 

Hint: you should maximize the entropy of $Y_i$ given that $\E(Y_i) = 2$.


\item We play a game. 
I flip a fair coin. 
If the coin shows head then you get your stake doubled with probability $0.5$ and you get nothing with probability $0.5$.
If the coin shows tail then you get your stake doubled with probability $0.9$ and you get nothing with probability $0.1$.

You maximize long-term interest rate. 
\begin{enumerate}
    \item How much of your current fortune your should invest if you make the stake \textit{before} the coin toss?
    \item How much of your current fortune your should invest if you make the stake \textit{after} the coin toss 
    and you know the result of the toss?
\end{enumerate}


\item The response variable is binary. 
Elon Musk minimizes the Gini impurity index and splits the node of a tree 
into two non-empty child nodes.

Provide an example for each case or prove that the case is not possible: 
\begin{enumerate}
    \item The impurity of both child nodes is lower than the impurity of the original node. 
    \item The impurity of the left child node is lower than the impurity of the original node,
    and the impurity of the right child node is higher. 
\end{enumerate}


\item Elon Musk forecasts the scalar random variable $y$ using the ensemble of $n$ models.
The model number $i$ provides its own forecast $\hat y_i$ with $\Corr(y, \hat y_i) = 0.1$.
Models are trained on random subsets of the initial dataset and hence their forecasts 
are correlated, $\Corr(\hat y_i, \hat y_j) = 0.2$ for $i \neq j$. 
The variances of the forecasts of each model are equal, $\Var(\hat y_1) = \Var(\hat y_2) = \ldots$ 

Elon Musk considers the ensemble forecast $\hat y = \sum \hat y_i / n$. 

\begin{enumerate}
    \item Find $\Corr(\hat y, y)$.
    \item Find the limit $\Corr(\hat y, y)$ when $n\to \infty$.
\end{enumerate}

Hint: please be careful, in this problem $i$ is the number of a model and not the number of an observation. 

\item Consider the naive bootstrap procedure. The initial sample consists of $n$ observations.
Consider the random variable $N_1$ — the number of copies of the first initial observation in a bootstrap sample.

Assume that the number of observations $n$ is very very big. 
\begin{enumerate}
    \item Find $\P(N_1 = 0)$ and $\P(N_1 = 1)$.
    \item Find the general formula $\P(N_1 = k)$. 
\end{enumerate}

Hint: you know the name of this distribution :)

\end{enumerate}


\end{document}

